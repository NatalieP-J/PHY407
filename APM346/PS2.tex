\documentclass[a4paper,12pt]{article}

\usepackage{rotating}
\usepackage[top=1in, bottom=1in, left=0.75in, right=0.75in]{geometry}
\usepackage{graphicx}
\usepackage[numbers,square,sort&compress]{natbib}
\usepackage{setspace}
\usepackage[cdot,mediumqspace,]{SIunits}
\usepackage{caption}
\usepackage{subcaption}
\usepackage{mathtools}
\usepackage{authblk}
\usepackage{float}
\usepackage{amsmath}
\renewcommand{\thesubsection}{\thesection.\alph{subsection}}
\providecommand{\e}[1]{\ensuremath{\times 10^{#1}}}
\newcommand*\Laplace{\mathop{}\!\mathbin\bigtriangleup}

\begin{document}
\onehalfspacing
\title{APM 346 Problem Set 4}
\author{Natalie Price-Jones, 999091021}
\date{24 October 2014}
\affil{\small{natalie.price.jones@mail.utoronto.ca}}
\maketitle

\section{Question 1}

Start with separation of variables $u(x,t) = X(x)T(t)$. The our PDE reduces to:

\begin{eqnarray}
Y\frac{d^2X}{dx^2} + X\frac{d^2Y}{dy^2} &=& 0\nonumber\\
\frac{1}{X}\frac{d^2X}{dx^2} + \frac{1}{Y}\frac{d^2Y}{dy^2} &=& 0\nonumber\\
\implies -\frac{1}{X}\frac{d^2X}{dx^2} = \frac{1}{Y}\frac{d^2Y}{dy^2} &=& -\lambda^2\nonumber
\end{eqnarray}

We assume as usual $\lambda > 0$. This must be true to satisfy boundary conditions and can be demonstrated in a way we have frequently done in class.

The ODEs we have found have the following solutions.

\begin{eqnarray}
Y(y) &=& A\sin(\lambda y) + B\cos(\lambda y)\nonumber\\
X(x) &=& Ce^{\lambda x} + De^{-\lambda x}\nonumber 
\end{eqnarray}

Now consider our boundary conditions: $Y(0) = Y(H) = 0$.

\begin{eqnarray}
\implies Y(0) &=& A\sin(\lambda(0)) + B\cos(\lambda(0)) = B = 0\nonumber\\
\mathrm{and}\: Y(H) &=& A\sin(\lambda H) = 0 \implies \lambda = \frac{n\pi}{H}\nonumber
\end{eqnarray}

Now apply boundary conditions on $X(L) = 0$.

\begin{eqnarray}
\implies X(L) &=& Ce^{\frac{n\pi}{H}L} + De^{-\frac{n\pi}{H}L} = 0\nonumber\\
\implies D &=& -Ce^{\frac{2n\pi}{H}L}\nonumber
\end{eqnarray}

So we have elementary solutions:

\begin{eqnarray}
u_n(x,y) = A_n\left(e^{\frac{n\pi}{H}x} - e^{\frac{2n\pi}{H}L}e^{-\frac{n\pi}{H}x}\right)\sin\left(\frac{n\pi}{H}y\right)\nonumber
\end{eqnarray}

From this we can compose a solution by making a linear combination of elementary solutions.

\begin{eqnarray}
u(x,y) = \sum_{n = 1}^{\infty}A_n\left(e^{\frac{n\pi}{H}x} - e^{\frac{n\pi}{H}(2L-1)}\right)\sin\left(\frac{n\pi}{H}y\right)\nonumber
\end{eqnarray}

Now apply final boundary condition to calculate the $A_n$'s. We want to satisfy $u(0,y) = f(y)$, so try to find $A_n$ such that:

\begin{eqnarray}
f(y) &=& \sum_{n = 1}^{\infty}A_n\left(1-e^{\frac{n\pi}{H}2L}\right)\sin\left(\frac{n\pi}{H}y\right)\nonumber\\
\int_{0}^{H}\sin\left(\frac{n'\pi}{H}y\right)f(y)\,dy &=& \sum_{n = 1}^{\infty}A_n\left(1-e^{\frac{n\pi}{H}2L}\right)\int_{0}^{H}\sin\left(\frac{n'\pi}{H}y\right)\sin\left(\frac{n\pi}{H}y\right)\,dy\nonumber\\
\int_0^H\sin\left(\frac{n'\pi}{H}y\right)f(y)\,dy &=& A_{n'}\left(1-e^{\frac{n'\pi}{H}2L}\right)\frac{H}{2}\nonumber\\
\implies A_{n'} &=& \frac{2}{H}\frac{1}{1-e^{\frac{n\pi}{H}2L}}\int_0^H\sin\left(\frac{n\pi}{H}y\right)f(y)\, dy\nonumber
\end{eqnarray}

Therefore we have a final answer in the form:

\begin{eqnarray}
u(x,y) = \frac{2}{H}\sum_{n=1}^{\infty}\left(\frac{e^{\frac{n\pi}{H}x} - e^{\frac{n\pi}{H}(2L-x)}}{1-e^{\frac{n\pi}{H}2L}}\right)\left(\int_{0}^{H}\sin\left(\frac{n\pi}{H}y\right)f(y)\,dy\right)\sin\left(\frac{n\pi}{H}y\right)\nonumber
\end{eqnarray}

\section{Question 2}

Prove that spherical Laplacian is equivalent to the three dimensional Carteisan Laplacian.

\begin{eqnarray}
L = \frac{\partial^2}{\partial r^2} + \frac{2}{r}\frac{\partial}{\partial r} + \frac{1}{r^2}\frac{\partial^2}{\partial \theta^2} + \frac{\cos\theta}{r^2\sin\theta}\frac{\partial}{\partial \theta} + \frac{1}{r^2 \sin^2\theta}\frac{\partial^2}{\partial\psi^2}\nonumber
\end{eqnarray}

Given equations for $(x,y,z)$, in terms of $(r,\theta,\psi)$:

\begin{eqnarray}
x &=& r\cos\psi\sin\theta \nonumber\\
y &=& r\sin\psi\sin\theta \nonumber\\
z &=& r\cos\theta\nonumber
\end{eqnarray}

We can rewrite our partial derivatives in terms of Cartesian coordinates:

\begin{eqnarray}
\frac{\partial}{\partial r} &=& \frac{\partial x}{\partial r}\frac{\partial}{\partial x} + \frac{\partial y}{\partial r}\frac{\partial}{\partial y} + \frac{\partial z}{\partial r}\frac{\partial}{\partial z}\nonumber\\
\frac{\partial^2}{\partial r^2} &=& \left(\frac{\partial x}{\partial r}\right)^2\frac{\partial^2}{\partial x^2} + \left(\frac{\partial y}{\partial r}\right)^2\frac{\partial^2}{\partial y^2} + \left(\frac{\partial z}{\partial r}\right)^2\frac{\partial^2}{\partial z^2} \nonumber\\
&+& 2\left(\frac{\partial x}{\partial r}\frac{\partial y}{\partial r}\frac{\partial^2}{\partial x \partial y} + \frac{\partial x}{\partial r}\frac{\partial z}{\partial r}\frac{\partial^2}{\partial x \partial z} + \frac{\partial y}{\partial r}\frac{\partial z}{\partial r}\frac{\partial^2}{\partial x \partial y}\right)\nonumber\\ 
&+& \frac{\partial x}{\partial r}\left(\frac{\partial}{\partial x}\left(\frac{\partial x}{\partial r}\right)\frac{\partial}{\partial x} + \frac{\partial}{\partial x}\left(\frac{\partial y}{\partial r}\right)\frac{\partial}{\partial y} + \frac{\partial}{\partial x}\left(\frac{\partial z}{\partial r}\right)\frac{\partial}{\partial z}\right)\nonumber\\ 
&+& \frac{\partial y}{\partial r}\left(\frac{\partial}{\partial y}\left(\frac{\partial y}{\partial r}\right)\frac{\partial}{\partial y} + \frac{\partial}{\partial y}\left(\frac{\partial x}{\partial r}\right)\frac{\partial}{\partial x} + \frac{\partial}{\partial y}\left(\frac{\partial z}{\partial r}\right)\frac{\partial}{\partial z}\right)\nonumber\\
&+& \frac{\partial z}{\partial r}\left(\frac{\partial}{\partial z}\left(\frac{\partial z}{\partial r}\right)\frac{\partial}{\partial z} + \frac{\partial}{\partial z}\left(\frac{\partial x}{\partial r}\right)\frac{\partial}{\partial x} + \frac{\partial}{\partial z}\left(\frac{\partial y}{\partial r}\right)\frac{\partial}{\partial y}\right)\nonumber
\end{eqnarray}

Where similar equations exist for $\theta$ and $\psi$

In the following derivation I use notation $\cos = c$ and $\sin = s$. We can find all of the partial derivatives in the preceding equation.

\begin{eqnarray}
\frac{\partial x}{\partial r} &=& c\psi s\theta,\: \frac{\partial y}{\partial r} = s\psi s\theta,\: \frac{\partial z}{\partial r} = c\theta\nonumber\\
\frac{\partial x}{\partial \theta} &=& rc\psi c\theta,\: \frac{\partial y}{\partial\theta} = r s\psi c\theta,\:\frac{\partial z}{\partial \theta} = -rs\theta\nonumber\\
\frac{\partial x}{\partial \psi} &=& -rs\psi s\theta,\: \frac{\partial y}{\partial \psi} = rc\psi s\theta,\:\frac{\partial z}{\partial \psi} = 0\nonumber
\end{eqnarray}

And further:

\begin{eqnarray}
\frac{\partial}{\partial x}\left(\frac{\partial x}{\partial r}\right) &=& \frac{1}{r} - \frac{x^2}{r^3},\:
\frac{\partial}{\partial y}\left(\frac{\partial y}{\partial r}\right) = \frac{1}{r} - \frac{y^2}{r^3},\:
\frac{\partial}{\partial z}\left(\frac{\partial z}{\partial r}\right) = \frac{1}{r} - \frac{z^2}{r^3}\nonumber\\
\frac{\partial}{\partial x}\left(\frac{\partial y}{\partial r}\right) &=& \frac{\partial}{\partial y}\left(\frac{\partial x}{\partial r}\right) = \frac{-xy}{r^3}\nonumber\\
\frac{\partial}{\partial z}\left(\frac{\partial x}{\partial r}\right) &=& \frac{\partial}{\partial x}\left(\frac{\partial z}{\partial r}\right) = \frac{-xz}{r^3}\nonumber\\
\frac{\partial}{\partial z}\left(\frac{\partial y}{\partial r}\right) &=& \frac{\partial}{\partial y}\left(\frac{\partial z}{\partial r}\right) = \frac{-yz}{r^3}\nonumber
\end{eqnarray}

If I had enough time, I would continue this derivation, finding appropriate expressions for the rest of the $\theta$ and $\psi$ terms.
However, I was unable to complete this problem set. Additionally, I do not feel that completeing this question will help in understanding the PDE material, though obviously I will finish it if I have time. The rest of this derivation is done as if only the first two lines of the second derivative terms were used, and obviously comes out to the wrong answer.

We rewrite the individual terms of $L$ in terms of the above derivatives:

\begin{eqnarray}
\frac{\partial^2}{\partial r^2} &=& c^2\psi c^2\theta \frac{\partial^2}{\partial x^2} + s^2\psi s^2\theta\frac{\partial^2}{\partial y^2} + c^2\theta\frac{\partial^2}{\partial z^2} \nonumber\\&+& 2\left( c\psi s\psi s^2\theta \frac{\partial^2}{\partial x \partial y} +  c\psi s\theta c\theta \frac{\partial^2}{\partial x \partial z} + s\psi s\theta c\theta \frac{\partial^2}{\partial y \partial z}\right)\nonumber\\
\frac{2}{r}\frac{\partial}{\partial r} &=& \frac{2}{r}\left(c\psi s\theta \frac{\partial}{\partial x} + s\psi s\theta\frac{\partial}{\partial y} + c\theta\frac{\partial}{\partial z}\right)\nonumber\\
\frac{1}{r^2}\frac{\partial^2}{\partial \theta^2} &=& c^2\psi c^2\theta \frac{\partial^2}{\partial x^2} + s^2\psi c^2\theta \frac{\partial^2}{\partial y^2} + s^2\theta\frac{\partial^2}{\partial z^2}\nonumber\\ &+& 2\left(c\psi s\psi c^2\theta \frac{\partial^2}{\partial x \partial y} - c\psi c\theta s\theta \frac{\partial^2}{\partial x \partial z} - s\psi c\theta s\theta \frac{\partial^2}{\partial y \partial z}\right)\nonumber\\
\frac{c\theta}{r^2 s\theta}\frac{\partial}{\partial \theta} &=& \frac{c\psi c^2\theta}{rs\theta} \frac{\partial}{\partial x} + \frac{s\psi c^2\theta}{r s\theta}\frac{\partial}{\partial y} - \frac{c\theta}{r}\frac{\partial}{\partial z}\nonumber\\
\frac{1}{r^2 s^2\theta}\frac{\partial^2}{\partial \psi^2} &=& s^2\psi \frac{\partial^2}{\partial x^2} + c^2\psi\frac{\partial^2}{\partial y^2} - 2\left(c\psi s\psi \frac{\partial^2}{\partial x \partial y}\right)\nonumber
\end{eqnarray}

Sum the above expressions and gather like terms in the derivatives:

\begin{eqnarray}
L &=& \left(c^2\psi c^2\theta + c^2\psi c^\theta + s\psi^2\right)\frac{\partial^2}{\partial x^2} + \left(s^2\psi s^2\theta + s^2\psi s^2\theta + c^2 \psi\right)\frac{\partial^2}{\partial y^2} + \left(c^2\theta + s^2\theta\right)\frac{\partial^2}{\partial z^2}\nonumber\\ &+& \left(2c\psi s\psi s^2\theta + 2c\psi s\psi c^2\theta - 2c\psi s\psi\right)\frac{\partial^2}{\partial x \partial y} + \left(2c\psi s\theta c\theta - 2c\psi c\theta s\theta \right)\frac{\partial^2}{\partial x \partial z} \nonumber\\&+& \left(2 s\psi s\theta c\theta - 2s\psi c\theta s\theta\right)\frac{\partial^2}{\partial y \partial z} + \left(\frac{2}{r}c\psi s\theta + \frac{c\psi c^2\theta}{r s\theta}\right)\frac{\partial}{\partial x} + \left(\frac{2}{r}s\psi s\theta + \frac{s\psi c^2\theta}{r s\theta}\right)\frac{\partial}{\partial y}\nonumber\\ &+& \left(\frac{2}{r}c\theta - \frac{c\theta}{r}\right)\frac{\partial}{\partial z}\nonumber
\end{eqnarray}

With copious applications of the identity $c^2a + s^2a = 1$, this reduces to :

\begin{eqnarray}
L = \frac{\partial^2}{\partial x^2} + \frac{\partial^2}{\partial y^2} + \frac{\partial^2}{\partial z^2} + \frac{c\psi}{r s\theta}\left(2 s^2\theta + c^2\theta\right)\frac{\partial}{\partial x} + \frac{s\psi}{r s\theta}\left(2s^2\theta + c^2\theta\right)\frac{\partial}{\partial x} + \frac{c\theta}{r}\frac{\partial}{\partial z}\nonumber
\end{eqnarray}

\section{Question 4}
\subsection{Part a)}
$D_au\equiv u(ax)$.

\begin{eqnarray}
\Laplace u_a = \sum_{i=1}^n \frac{\partial^2}{\partial x_i^2}u(ax) = \sum_{i=1}^n a^2 \left(\frac{\partial^2}{\partial x_i^2}u(x)\right)_a = a^2(\Laplace u)_a
\end{eqnarray}

Now want to find $\mathcal{D}$ such that $\frac{d}{da}\big|_{a=1} = \mathcal{D}$. Try this by applying the left side to a function $u$.

\begin{eqnarray}
\frac{d}{da}\Big|_{a=1}u = \frac{\partial x_*}{\partial a}\nabla_{*}u(ax)\Big|_{a=1} = x\nabla_* u(ax)\Big|_{a=1} = x\nabla u(x)\nonumber
\end{eqnarray}

Where in the above $\nabla_{*}$ denotes the gradient with respect to $x_* = ax$. So $\mathcal{D} = x\nabla$.

\subsection{Part b)}

\begin{eqnarray}
\Laplace(T_y u) = \sum_{i=1}^n \frac{\partial^2}{\partial x_i^2}(u(x-y)) = \sum_{i=1}^n \frac{\partial^2u(x-y)}{\partial x_i^2} = \sum_{i=1}^n T_y\frac{\partial^2u(x)}{\partial x_i^2} = T_y\Laplace u\nonumber
\end{eqnarray}

Where we have used the fact that $\frac{\partial}{\partial x_i} (x-y) = \frac{\partial}{\partial x_i}x$ since $y$ is a constant.

\subsection{Part c)}

Start with:

\begin{eqnarray}
\nabla (u\circ A) &=&  A^T\nabla u\nonumber\\
\Laplace (u\circ A) &=& \nabla \cdot \nabla(u\circ A)\nonumber\\
\Laplace (u\circ A) &=& \nabla \cot A^T\nabla u\nonumber\\
\Laplace (u\circ A) &=& \Laplace u \circ A
\end{eqnarray}

\subsection{Part d)}

Use the results of the previous three parts to determine the commutators:

Start with $[\Laplace,D_a]$ and evaluate it by first applying it to a function $u$
\begin{eqnarray}
[\Laplace,D_a]u &=& \Laplace(D_a(u)) - D_a(\Laplace(u)) = a^2 D_a(\Laplace(u)) - D_a\Laplace(u) = (a^2 - 1)D_a(\Laplace(u))\nonumber\\
\left[\Laplace,T_y\right]u &=& \Laplace(T_y(u)) - T_y(\Laplace(u)) = T_y(\Laplace(u)) - T_y(\Laplace(T_y(u))) = 0 \nonumber\\
\left[\Laplace,R_A\right]u &=& \Laplace(R_A(u)) - R_A(\Laplace(u)) = \Laplace (u\circ A) - (\Laplace(u))\circ A = \Laplace(u\circ A) - \Laplace(u\circ A) = 0\nonumber
\end{eqnarray}

So we find that we can write the commutators as the following operators:

\begin{eqnarray}
[\Laplace,D_a] &=& (a^2 - 1)D_a\Laplace\nonumber\\
\left[\Laplace,T_y\right] &=& 0\nonumber\\
\left[\Laplace,R_A\right] &=& 0\nonumber
\end{eqnarray}

\end{document}





